\documentclass[twocolumn, a4paper]{UECIEresume}

\usepackage[dvipdfmx]{graphicx}
\usepackage{graphicx}
\usepackage{amsmath}
\usepackage{txfonts}

\title{卒論,修論予稿フォーマット}
\date{平成 29 年 2 月 14 日}
\affiliation{総合情報学科 メディア情報学 コース}
\supervisor{橋山智訓 准教授}
\studentid{1310163}
\author{柴澤弘樹}
%\headtitle{平成 yy 年度 総合情報学科 卒業論文中間発表}
\headtitle{平成 28 年度 総合情報学科 卒業論文発表}
%\headtitle{平成 yy 年度 総合情報学科 修士論文中間発表}
%\headtitle{平成 yy 年度 総合情報学科 修士論文発表}

\begin{document}
\maketitle

\section{序論}


\section{DQNによる戦術の学習}


\section{人の知識を適用したAI}

\section{ゲーム内AIとの対戦}


\section{結論}
最後は簡潔に研究成果をまとめて下さい.将来の課題などもあれば書いても良いですが,あまり課題を書きすぎると逆効果になりますのでほどほどにしておきましょう.

また,引用文献はキチンと入れましょう\cite{Kinoshita1}.引用は,先人に対するリスペクトなので,よほど独立性が高い研究でない限り必要となります.

{\small
\begin{thebibliography}{*}
\bibitem{Kinoshita1} 木下 是雄, \textit{理科系の作文技術}, 中公新書 624, 1981.
\end{thebibliography}
}
\end{document}
